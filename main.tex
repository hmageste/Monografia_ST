% ------------------------------------------------------------------------
% ------------------------------------------------------------------------
% abnTeX2: Modelo de Trabalho Academico (tese de doutorado, dissertacao de
% mestrado e trabalhos monograficos em geral) em conformidade com 
% ABNT NBR 14724:2011: Informacao e documentacao - Trabalhos academicos -
% Apresentacao
% ------------------------------------------------------------------------
% ------------------------------------------------------------------------
\documentclass[
	% -- opções da classe memoir --
	12pt,						% tamanho da fonte
	openright,					% capítulos começam em pág ímpar (insere página vazia caso preciso)
	twoside,					% para impressão em verso e anverso. Oposto a oneside
	a4paper,					% tamanho do papel. 
				% -- opções da classe abntex2 --
	%chapter=TITLE,			% títulos de capítulos convertidos em letras maiúsculas
	%section=TITLE,			% títulos de seções convertidos em letras maiúsculas
	%subsection=TITLE,			% títulos de subseções convertidos em letras maiúsculas
	%subsubsection=TITLE,		% títulos de subsubseções convertidos em letras maiúsculas
				% -- opções do pacote babel --
	english,					% idioma adicional para hifenização
	spanish,					% idioma adicional para hifenização
	brazil					% o último idioma é o principal do documento
	]{abntex2}

% ---
% Pacotes básicos 
% ---
\usepackage{lmodern}			% Usa a fonte Latin Modern			
\usepackage[T1]{fontenc}		% Selecao de codigos de fonte.
\usepackage[utf8]{inputenc}		% Codificacao do documento (conversão automática dos acentos)
\usepackage{lastpage}			% Usado pela Ficha catalográfica
\usepackage{indentfirst}		% Indenta o primeiro parágrafo de cada seção.
\usepackage{color}				% Controle das cores
\usepackage{graphicx}			% Inclusão de gráficos
\usepackage{microtype} 			% para melhorias de justificação

\usepackage{float}				% include H option in figures
\graphicspath{ {./figures/} }	% Give the relative-path where images are stored

\usepackage{subfiles}

% ---
% Pacotes de citações
% ---
\usepackage[brazilian,hyperpageref]{backref}	 % Paginas com as citações na bibl
\usepackage[alf]{abntex2cite}		% Citações padrão ABNT

\usepackage{mystyle}

%\usepackage[backend=bibtex]{biblatex}
%\addbibresource{references.bib}

% ---
% Informações de dados para CAPA e FOLHA DE ROSTO
% ---
\titulo{Avaliação das Condições de Segurança do Trabalho e Saúde Ocupacional em um Navio de Contenção de Óleo}
\autor{Henrique Mageste}
\local{Rio de Janeiro, RJ - Brasil}
\data{\today}
\orientador{Justino Sanson Wanderley da Nóbrega, M. Sc.}
%\coorientador{Equipe \abnTeX}
\instituicao{
  Universidade Federal do Rio de Janeiro - UFRJ
  \par
  Engenharia de Segurança do Trabalho
  \par
  Programa de Pós-Graduação}
\tipotrabalho{Monografia}
\preambulo{Monografia Submetida ao Corpo Docente do Curso de Pós-Graduação em Engenharia de Segurança do Trabalho
da Universidade Federal do Rio de Janeiro como Parte dos Requisitos Necessários para Obtenção do Título de Especialista
em Engenharia de Segurança do Trabalho}
% ---

\makeindex

\begin{document}

\selectlanguage{brazil}
\frenchspacing					% Retira espaço extra obsoleto entre as frases

% ----------------------------------------------------------
% ELEMENTOS PRÉ-TEXTUAIS
% ----------------------------------------------------------
%\pretextual ou %\frontmatter

\imprimircapa
\imprimirfolhaderosto*		% o * indica que haverá a ficha bibliográfica

\input{include/0-fichacatalografica}
\input{include/0-folhadeaprovacao}
\begin{dedicatoria}
   \vspace*{\fill}
   \centering
   \noindent
   \textit{Este trabalho é dedicado a todos os engenheiros,\\
	   de segurança ou não, que queiram se aventurar na\\
	   legislação naval.} \vspace*{\fill}
\end{dedicatoria}
\begin{agradecimentos}
Os agradecimentos principais são direcionados ao professor Justino da Nóbrega,
pela sua paciência e orientação durante todo o curso deste trabalho.

Agradecimentos especiais são direcionados à empresa e à sua coordenadora ambiental que
possibilitaram o desenvolvimento deste trabalho.

Aos professores do curso de Pós-Graduação em Engenharia de Segurança do Trabalho da UFRJ,
pelos ensinamentos proporcionados.

À minha namorada, Alexandra, e aos amigos que direta ou indiretamente contribuíram para
minha formação profissional e pessoal.

\end{agradecimentos}
\begin{epigrafe}
    \vspace*{\fill}
	\begin{flushright}
		\textit
		{
		    ``Se enxerguei mais longe, \\
		    foi porque me apoiei sobre os ombros de gigantes''\\
		    (Isaac Newton)
		}
	\end{flushright}
\end{epigrafe}
\input{include/0-summary}

% ---
% inserir lista de ilustrações
% ---
\pdfbookmark[0]{\listfigurename}{lof}
\listoffigures*
\cleardoublepage
% ---

% ---
% inserir lista de tabelas
% ---
\pdfbookmark[0]{\listtablename}{lot}
\listoftables*
\cleardoublepage
% ---

% ---
% inserir lista de abreviaturas e siglas
% ---
\begin{siglas}
  \item[ABNT] Associação Brasileira de Normas Técnicas
  \item[APR] Análise Preliminar de Risco
  \item[ASO] Atestado de Saúde Ocupacional
  \item[CA] Certificado de Aprovação
  \item[CIPA] Comissão Interna de Prevenção de Acidentes
  \item[CLT] Consolidação das Leis do Trabalho
  \item[DRT] Delegacia Regional de Trabalho
  \item[EPC] Equipamento de Proteção Coletiva
  \item[EPI] Equipamento de Proteção Individual
  \item[ISO] \emph{International Organization for Standardization}
  \item[LT] Limite de Tolerância
  \item[MTE] Ministério de Trabalho e Emprego
  \item[NBR] Norma Técnica Brasileira
  \item[NR] Norma Regulamentadora
  \item[OHSAS] \emph{Occupational Health and Safety Assessment Series}
  \item[PCMSO] Programa de Controle Médico de Saúde Ocupacional
  \item[PET] Permissão da Entrada de Trabalho
  \item[PPRA] Programa de Prevenção de Riscos Ocupacionais
  \item[PT] Permissão de Trabalho
  \item[SIPAT] Semana Interna de Prevenção de Acidentes de Trabalho
  \item[SMS] Segurança, Meio Ambiente e Saúde
  \item[SSO] Segurança e Saúde Ocupacional
\end{siglas}
% ---

% ---
% inserir lista de símbolos
% ---
\begin{simbolos}
  \item[$ \Gamma $] Letra grega Gama
  \item[$ \Lambda $] Lambda
  \item[$ \zeta $] Letra grega minúscula zeta
  \item[$ \in $] Pertence
\end{simbolos}
% ---

% ---
% inserir o sumario
% ---
\pdfbookmark[0]{\contentsname}{toc}
\tableofcontents*
\cleardoublepage
% ---

% ----------------------------------------------------------
% ELEMENTOS TEXTUAIS
% ----------------------------------------------------------
\textual

\subfile{include/1-introduction}
\subfile{include/2-embasement}
\subfile{include/2-studysubject}
\subfile{include/3-NR_analysis}

\phantompart % Finaliza a parte no bookmark do PDF para que se inicie o bookmark na raiz e adiciona espaço de parte no Sumário
%\include{include/5-conclusion}

% ----------------------------------------------------------
% ELEMENTOS PÓS-TEXTUAIS
% ----------------------------------------------------------
\postextual

\bibliography{bibliography}

%\glossary		% Consulte o manual da classe abntex2 para orientações sobre o glossário.

%% ----------------------------------------------------------
% Apêndices
% ----------------------------------------------------------
\begin{apendicesenv}
\partapendices		% Imprime uma página indicando o início dos apêndices

\chapter{Apendice A}

\end{apendicesenv}

% ----------------------------------------------------------
% Anexos
% ----------------------------------------------------------
\begin{anexosenv}
\partanexos		% Imprime uma página indicando o início dos anexos

\chapter{Anexo A}

\end{anexosenv}


%---------------------------------------------------------------------
% INDICE REMISSIVO
%---------------------------------------------------------------------
\phantompart
\printindex
%---------------------------------------------------------------------

\end{document}
