%-------------------------------- Configura��es --------------------------------

\documentclass[a4paper,         % Tamanho do papel: A4
                 abntfigtabnum,
                 noindentfirst,
                 normaltoc,
                 pnumplain,
                 notimes
                 % capchap,
]{abnt}

% Links border color
\newcommand{\bc}{NavyBlue}
\usepackage{booktabs}% http://ctan.org/pkg/booktabs
\newcommand{\tabitem}{~~\llap{\textbullet}~~}
\usepackage[utf8]{inputenc} % para pode escrever caracteres acentuados normalmente
\usepackage[brazil]{babel}
\usepackage{graphicx}
\usepackage[usenames,dvipsnames]{xcolor} % http://en.wikibooks.org/wiki/LaTeX/Colors
\usepackage[pdfborder={0 0 0},pdfborderstyle={/S/U/W 0.5},citebordercolor=\bc,filebordercolor=\bc,urlbordercolor=\bc,linkbordercolor=\bc]{hyperref} % http://www.tug.org/applications/hyperref/manual.html e http://migre.me/7FH3e
\usepackage[alf]{abntcite}
\usepackage{textcomp}
\usepackage{lineno}
\usepackage{nomencl} % para criar a lista de siglas
\usepackage{lscape} % para usar p�ginas em landscape (deitadas)
\usepackage{hyperref}
\usepackage{caption}
\usepackage{minted}

\renewcommand\listingscaption{C�digo}

%------------------------- Numerar subsubsection -------------------------------

\makeatletter
\newcommand\sparagraph{\@startsection{section}{1}{\z@}%
                                    {3.25ex \@plus1ex \@minus.2ex}%
                                    {-1em}%
                                    {\normalfont\normalsize\bfseries}}
\newcommand\ssparagraph{\@startsection{subsection}{2}{\z@}%
                                    {3.25ex \@plus1ex \@minus.2ex}%
                                    {-1em}%
                                    {\normalfont\normalsize\bfseries}}
\newcommand\sssparagraph{\@startsection{subsubsection}{3}{\z@}%
                                    {3.25ex \@plus1ex \@minus.2ex}%
                                    {-1em}%
                                    {\normalfont\normalsize\bfseries}}
\setcounter{secnumdepth}{3}% Allow numbering up to \subsubsection or \sssparagraph
\makeatother

%----------------------- Quebra de linha ap�s paragraph ------------------------

\makeatletter
\renewcommand\paragraph{\@startsection{paragraph}{4}{\z@}%
  {-3.25ex\@plus -1ex \@minus -.2ex}%
  {1.5ex \@plus .2ex}%
  {\normalfont\normalsize\bfseries}}
\makeatother

%--------------------------------- Informa��es ---------------------------------

\newcommand{\meutitulo}{Modelagem de cloud computing para gerenciamento de recursos computacionais na UENF}

% http://www.tug.org/applications/hyperref/manual.html
\hypersetup{
  pdftitle=\meutitulo,
  pdfauthor=Douglas Oliveira Camata
}

\makenomenclature

\begin{document}

  \titulo{\meutitulo}
  \autor{Douglas Oliveira Camata}
  \instituicao{Universidade Estadual do Norte Fluminense Darcy Ribeiro\par Laborat�rio de Ci�ncias Matem�ticas}
  \orientador[Orientadora:\par]{Prof�. Dr�. Annabell del Real Tamariz}
  \comentario{Monografia apresentada ao curso de gradua��o em Ci�ncia da Computa��o da Universidade Estadual do Norte Fluminense Darcy Ribeiro como requisito para obten��o do t�tulo de Bacharel em Ci�ncia da Computa��o.}
  \local{Rio de Janeiro - RJ}
  \data{2014}

  \capa
  \folhaderosto

%  \input{tex/folha_de_aprovacao}
%  % \input{tex/epigrafe}
%  \input{tex/agradecimentos}
%  \input{tex/resumo}
%  \input{tex/abstract}

  \sumario
  \listoffigures
  \listoftables
  \printnomenclature

%  \input{tex/introducao}
%  \input{tex/fundamentacao}
%  \input{tex/estado_da_arte}
%  \input{tex/experimento  }
%  \input{tex/resultados}
%  \input{tex/conclusoes}

  % \anexo
  % \input{tex/anexo_efetividade_tdd}
  % \input{tex/anexo_codigos_do_comparativo}

  %--------------------------------- Bibliografia ------------------------------

  \citeoption{abnt-repeated-author-omit=yes}
  \bibliographystyle{abnt-alf}
  \bibliography{bibliografia}
\end{document}