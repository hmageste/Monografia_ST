\chapter{Análise das NR}
\section{NR-1 - Disposições Gerais}
Esta é NR que trata das disposições gerais sobre a observância obrigatória das Normas Regulamentadoras (NR) pelas empresas privadas e públicas e pelos órgãos públicos da administração direta e indireta bem como pelos órgãos dos Poderes Legislativo e Judiciário, que possuam empregados regidos pela Consolidação das Leis do Trabalho - CLT.

Segundo ao item 1.2 desta norma:

\begin{citacao}
A observância das NR não desobriga as empresas do cumprimento de outras disposições que, com relação à matéria, sejam incluídas em códigos de obras ou regulamentos sanitários dos Estados ou Municípios, e outras, oriundas de convenções e acordos coletivos de trabalho.
\end{citacao}

Portanto, para a completa análise do objeto de estudo deste trabalho, recorreremos ao \emph{International Safety Management Code} (\emph{ISM Code}) ou, em português, Código Internacional de Gerenciamento para Operação Segura e para a Prevenção da Poluição, sempre que necessário.

\section{NR-2 - Inspeção Prévia}
A NR-2 trata da solicitação de licensa prévia para funcionamento instalação antes desta iniciar suas atividades. Esta licensa deve ser solicitada junto ao Ministério de Trabalho e Emprego como estipulado nesta mesma NR.

Considerando que o objeto de estudo deste trabalho não é a empresa em si, mas sim uma embarcação pertencente a esta, torna-se necessário analisar a documentação pertinente para a operação segura da embarcação em suas atividades. Esse documento é o Certificado de Gerenciamento de Segurança exigido pelo Código Internacional de Gerenciamento para Operação Segura e para a Prevenção da Poluição (ISM Code). Esse documento certifica que o sistema de segurança do navio foi submetido a uma auditoria e que ele atende aos requisitos deste código e, ainda, que foi verificado que o Documento de Conformidade da Companhia é aplicável a este tipo de navio.

Sendo assim, após análise do certificado da companhia, verificou-se que o navio em estudo foi submetido à esta auditoria e atendeu aos requisitos do Código ISM.

\section{NR-3 - Embargo ou Interdição}
A NR-3 trata sobre o embargo ou interdição a partir da constatação de situação de trabalho que caracterize risco grave e iminente ao trabalhador. Para esta norma, segundo seu item 3.1.1, essa caracterização é dada por:

\begin{citacao}
Considera-se grave e iminente risco toda condição ou situação de trabalho que possa causar acidente ou doença relacionada ao trabalho com lesão grave à integridade física do trabalhador.
\end{citacao}

O \emph{ISM Code} não trata exatamente sobre embargo ou interdição de uma atividade, mas diz, em seu capítulo 8, que a empresa deve estabelecer procedimentos para caso acidentes ocorram durante essas atividades.

Verificou-se que a empresa possui um plano de emergência para este tipo de ocorrência.
