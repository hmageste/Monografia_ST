% ----------------------------------------------------------
% Introdução (capítulo sem numeração, mas presente no Sumário)
% ----------------------------------------------------------
\chapter*[Introdução]{Introdução}
\addcontentsline{toc}{chapter}{Introdução}
% ----------------------------------------------------------

Ao refletirmos sobre o modelo de desenvolvimento que proporcionou o rápido crescimento industrial das sociedades capitalistas até o início deste século, é possível observar uma lógica voltada exclusivamente para o lucro, dando-se pouca importância para a qualidade da saúde e segurança dos trabalhadores que neste meio exerciam suas atividades. Nota-se também, nesta época, a falta de uma política ambiental sustentável.

Questões como essas, atualmente, são foco de debates por diversas organizações mundiais em busca de soluções ditas como sustentáveis. Em parte, devido ao grande crescimento populacional ocorrido desde o início deste século, verificou-se a necessidade de se pensar em estratégias que causassem menores impactos ao ambiente ao mesmo tempo que aumentassem a produção exigida pela enorme demanda populacional, sendo este raciocínio o pilar da teoria do desenvolvimento sustentável.

Políticas de proteção ao meio ambiente e melhorias nos processos produtivos começaram a surgir por todas as partes do mundo devido a essa nova necessidade de tornar os processos produtivos em processos menos danosos ao ambiente e a sociedade para qual essa produção seria destinada. Dentre essas políticas, uma que chamou muito a atenção foi quanto a saúde e segurança do trabalhador em seu ambiente de trabalho, já que ele, o trabalhador, era o responsável direto pelo processo produtivo e principalmente um consumidor em potencial.

Como nem todas as políticas adotadas por esses países apresentavam os mesmos requisitos, certificações internacionais foram criadas a fim de mitigar as divergências encontradas entre as políticas internas adotadas por esses países e proporcionar uma referência para a criação de novas políticas de segurança, meio ambiente e saúde (SMS). Dentre essas certificações, estão a ISO 14000:2004 e OHSAS 18000:2007, responsáveis por implementar um sistema de gestão nas áreas de meio ambiente e saúde através de uma série de requisitos que, aliados a uma ideia de melhoria contínua, visam diminuir ou acabar com os riscos nessas áreas.

Um ponto extremamente importante é que obtenção destas certificações sem o comprometimento da empresa quanto à implementação de um sistema com melhoria contínua foge ao escopo e propósito buscados pelas certificações em questão, já que proporcionam a não mitigação dos riscos não tratados pelas diretrizes destas, permanecendo estes ocultos e passíveis de danos ao trabalhador, meio ambiente e a sociedade em geral.

Além dos já mencionados sistemas de gestão, de âmbito internacional, existem as Normas Regulamentadoras (NR), que foram aprovadas pelo Ministério do Trabalho e Emprego (MTE), e portanto de âmbito nacional, pela Portaria 3.214 de 08 de Junho de 1978. Estas normas estabelecem os requisitos mínimos nos aspectos técnicos legais sobre Segurança e Saúde Ocupacional (SSO) brasileira objetivando a garantia da saúde do trabalhador em sua atividade laboral e a implementação de políticas que minimizem riscos e melhorem o desempenho de suas atividades.

Todas as empresas os instituições que tenha empregados regidos pela Consolidação das Leis do Trabalho (CLT) têm como obrigatoriedade a observância das NR. Em conjunto as estas, ainda existem outras obrigatoriedades a serem atendidas quanto aos requisitos legais de segurança do trabalho e saúde ocupacional, tais como leis, decretos, portarias, medidas provisórias e etc.

\section{Objetivos do Estudo}
\subsection{Objetivo Geral}
Tendo base que o atendimento desses requisitos legais são itens básicos para garantir a saúde e segurança do trabalhador. Este documento tem por objetivo analisar o atendimento das condições de segurança e saúde ocupacional descritos pelas Normas Regulamentadoras do Ministério do Trabalho e Emprego para um navio de contenção de derramamento de óleo e a análise da legislação aplicável a embarcações do porte da embarcação em estudo.
\subsection{Objetivo Específico}
É esperado, ao final deste estudo, a identificação de possíveis melhorias nas condições de segurança e saúde ocupacionais, propondo um plano de ação com soluções adequadas para assegurar um sistema de trabalho seguro e saudável para os colaboradores.
Em paralelo, serão destacados possíveis não-conformidades e valores das multas pelo não atendimento aos itens das Normas Regulamentadoras com o intuito de demonstrar a necessidade de um investimento por parte da empresa na adequação destas e consequente melhoramento do ambiente laboral para os trabalhadores.

\section{Justificativa}

\section{Descrição do Trabalho}
Este trabalho é estruturado nos seguintes capítulos:
Capítulo 2, "Apresentação da Empresa", apresenta-se a empresa que será estudada bem como a embarcação que será o foco deste estudo.
Capítulo 3, "Metodologia", apresentam-se as normas utilizadas como base para analisar a embarcação em questão bem como a  metodologia utilizada para a análise do atendimento aos itens dessas normas. 
Capítulo 4, "Análise das Normas Regulamentadoras", apresenta-se o estudo do atendimento da embarcação às normas de referência adotadas. 
Capítulo 5, "Considerações Finais",  apresentam-se as conclusões e sugestões de melhorias a serem adotadas pela empresa/embarcação quanto ao atendimento das normas e possíveis melhorias no sistema de trabalho.

