\documentclass[../main.tex]{subfiles}
\begin{document}

\chapter{Embasamento}
Este capítulo objetiva dar ao leitor o embasamento para o entendimento dos assuntos normativos que serão tratados nesse trabalho. Portanto, sua leitura não é necessária para leitores que já estejam familiarizados com os tópicos aqui discutidos.

\section{Convenção SOLAS} \label{sec:SOLAS_convention}
 A Convenção Internacional para a Salvaguarda da Vida Humana no Mar ou, simplesmente, Convenção SOLAS (\emph{Safety of Life at Sea}), nas suas formas sucessivas, é geralmente considerada como o mais importante de todos os tratados internacionais sobre a segurança dos navios mercantes.
 
 O objectivo principal da Convenção consiste em especificar padrões mínimos para a construção, equipamento e operação de navios, compatíveis com a sua segurança. Os Estados são responsáveis por garantir que os navios sob a sua bandeira cumpram as suas exigências.
 
 O naufrágio do Titanic em 14 de Abril de 1912, depois de colidir com um iceberg, foi o catalisador para a adoção em 1914 desta convenção. Mais de 1.500 passageiros e tripulantes morreram e o desastre levantou muitas perguntas sobre as normas de segurança em vigor, pelo que o Governo do Reino Unido propôs a realização de uma conferência para elaborar regulamentos internacionais. A conferência, que contou com a presença de representantes de 13 países, introduziu novos requisitos internacionais relacionados com a segurança da navegação de todos os navios mercantes.

 A primeira versão foi aprovada em 1914, a segunda em 1929, a terceira em 1948 e a quarta em 1960. A versão de 1974 inclui o procedimento de aceitação tácita que prevê que uma alteração entra em vigor na data especificada, a menos que, antes dessa data, as objecções à emenda sejam recebidas por um número acordado de partes. Como resultado, a Convenção de 1974 foi atualizada e alterada em várias ocasiões. A Convenção em vigor, hoje, é por vezes referida como SOLAS de 1974, conforme alterada.
 
 A atual Convenção SOLAS inclui os artigos que estabelecem obrigações gerais, procedimento de alteração e assim por diante, seguidos por um anexo dividido em 12 capítulos.
 
\section{Código ISM} \label{sec:ISM-code}
 O Código ISM ou Código Internacional da Gestão da Segurança (em inglês \emph{International Safety Managment Code}) é uma tentativa de estabelecer um padrão internacional para a operação e gerenciamento seguros de navios e para a prevenção da poluição.
 
 Um fato importante a ser salientado é que a adoção do Código ISM é compulsória, ou seja, seu emprego é obrigatório de acordo com o capítulo IX da Convenção Internacional para a Salvaguarda da Vida Humana no Mar - SOLAS (Ver Seção \ref{sec:SOLAS_convention}).
 
 A iniciativa de se estabelecer padrões para a segurança no mar advém de que cerca de 80\% dos acidentes envolvendo navios são resultados diretos de erros humanos, não sendo considerado neste cálculo a parcela relacionada aos efeitos humanos indiretos. Paradoxalmente, as convenções anteriores ao Código ISM se debruçavam em aspectos técnicos dos navios e seus equipamentos, deixando de lado as razões, significativamente comprovadas, de ocorrência de acidentes.
 
 \subsection{A Origem}
  Foi em Outubro de 1989, na sequência de grandes acidentes marítimos, que a Organização Internacional Marítima (IMO - \emph{International Maritime Organization}), aprovou uma resolução, com orientações sobre a gestão para a segurança da exploração dos navios e a prevenção da poluição. Pretendia-se com esta resolução, fornecer aos responsáveis pela operação de navios, uma boa estrutura para o desenvolvimento, implementação e avaliação da segurança e gestão da prevenção da poluição.
  
  Em 1993, depois de alguma experiência no uso das diretrizes, a IMO adotou o Código ISM. Quatro anos depois, em 1997, a IMO adotou uma resolução que define a sua visão, princípios e metas para o elemento humano. Nesta resolução, ficou claramente evidenciado que o elemento humano é uma questão multi-dimensional complexa, que afeta a segurança marítima, a segurança e a proteção do meio ambiente marinho,
envolvendo todo o espectro das atividades humanas.

  No entanto, foi em 1998, que o Código ISM tornou-se obrigatório, de acordo com as disposições do
Capítulo IX, da Convenção SOLAS, tendo sido sucessivamente atualizado por várias emendas. A última delas, realizada em Junho de 2013, entrou em vigor em primeiro de Janeiro de 2015.  
  
  A motivação da IMO ao criar o Código ISM era de que as falhas nos procedimentos de bordo eram resultados de falhas nos procedimentos de gestão da companhia e sanar estas falhas significaria estabelecer padrões comparativos objetivando a identificação destas no sistema de gestão adotado pela companhia detentora do navio.
  
  Apenas para exemplificar o exposto, há uma grande probabilidade de um problema técnico ocorrido em um navio ser resolvido por meio de sua tripulação e não haver nenhum registro documentando este problema ou mesmo evidência da intervenção da empresa detentora do navio. Em outras palavras, não haveria evidências deste problema se o mesmo ocorresse novamente e, por conseguinte, nenhum meio de mitigação.
  
 \subsection{Estrutura}
  O Código ISM é baseado em princípios gerais e objetivos para abranger as diversas condições de operação de navios. Segundo \citeonline{sardinha2013}, efetivamente, o que mais conta para o resultado final, em matéria de segurança e prevenção da poluição, são as atitudes, o compromisso, a competência e a motivação dos indivíduos e das equipes envolvidas em todos os níveis do processo de implementação, mantenimento e regulação do código.
  \subsubsection{Aplicação}
   De acordo com a convenção SOLAS, o Código ISM se aplica a navios, independente de sua data de construção da seguinte forma:
   \begin{itemize}
    \item Navios de passageiros, inclusive embarcações de passageiros de alta velocidade;
    \item Petroleiros, navios de produtos químicos, navios transportadores de gás, graneleiros e embarcações de transporte de carga de alta velocidade, de arqueação bruta igual a 500GRT ou mais;
    \item Outros navios de carga e unidades móveis de perfuração marítima com arqueação bruta igual a 500GT ou mais.
   \end{itemize}
  
   O Código ISM não se aplica a navios operados por governos, utilizados para fins não comerciais.
  
  \subsubsection{Requisitos para a Gestão de Segurança}
   \begin{itemize}
    \item As companhias e os navios deverão cumprir as exigências do Código ISM, devendo os seus requisitos serem tratados como obrigatórios;
    \item Os navios deverão ser operados por companhias que possuam um Documento de Conformidade. 
   \end{itemize}
  
  \subsubsection{Certificação}
   \begin{itemize}
    \item Deverá ser emitido um Documento de Conformidade a todas as companhias que cumpram as exigências do Código Internacional de Gestão da Segurança. Este documento deverá ser emitido pela Administração, por uma organização reconhecida pela Administração ou, mediante solicitação da Administração, por outro Governo Contratante;
    \item Deverá ser mantido a bordo do navio uma cópia do Documento de Conformidade de modo que o comandante possa exibi-lo, quanto solicitado para verificação;
    \item Será emitido para cada navio, pela Administração ou por uma organização reconhecida pela Administração, um Certificado denominado Certificado de Gestão de Segurança;
    \item Antes de emitir o Certificado de Gestão de Segurança, a Administração ou a organização reconhecida por ela, verificará se a companhia e a gestão realizada a bordo de navios está de acordo com o sistema de gestão de segurança aprovado. 
   \end{itemize}
  \subsubsection{Verificação e Controle}
   A Administração, ou outro Governo Contratante, mediante solicitação da Administração ou uma organização reconhecida pela Administração, deverá verificar periodicamente, o funcionamento apropriado do sistema de gestão de segurança do navio.
\end{document}