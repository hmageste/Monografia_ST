\chapter{Objeto de Estudo}
O objeto de estudo deste trabalho, como mencionado anteriormente é um navio de contenção de óleo, referido neste trabalho como N-OCP. 

Embora a empresa proprietária deste navio, referida neste trabalho como OCP, não seja o objeto de análise deste trabalho, por motivos de estruturação de ideias uma pequena a apresentação do escopo de seu trabalho e quantitativo de funcionários se faz necessário.

\section{Caracterização da Empresa}
A OCP é uma empresa brasileira dedicada ao gerenciamento e resposta a emergências e ao desenvolvimento e implantação de soluções ligadas ao meio ambiente marinho e costeiro, principalmente para as indústrias marítima, portuária, pesqueira e de óleo e gás.
A OCP tem como principais atividades:
\begin{itemize}
\item Consultoria
\item Treinamento (com acreditação pelo emph{Nautical Institute})
\item Afretamento e operação de embarcações
\item Gerenciamento e resposta a emergências
\item Fornecimento de tripulações especializadas
\item \emph{Marine Survey} (oceanografia, geofísica e geotécnica)
\end{itemize}

No que diz respeito ao item Gerenciamento e resposta a emergências, a empresa conta com um enorme variedade de navios e barcos destinados a tal fim, sendo um deles o objeto de análise do presente estudo.

A embarcação escolhida para análise é a mais antiga da frota. A motivação para esta escolha se deve ao vasto histórico de não-conformidades quanto a fatores de segurança do trabalho evidenciados no decorrer das atividades desta embarcação desde seu primeiro contrato. 

\section{Caracterização do Embarcação}
