\chapter{Objeto de Estudo}
O objeto de estudo deste trabalho, como mencionado anteriormente, é um navio de contenção de óleo, aqui chamado de N-OCP. 

Embora a empresa proprietária deste navio, referida como OCP, não seja diretamente o objeto de análise deste trabalho, por motivos de estruturação de ideias uma pequena apresentação do escopo de seu trabalho e quantitativo de funcionários se fazem necessários.

\section{Caracterização da Empresa}
A OCP é uma empresa brasileira dedicada ao gerenciamento e resposta a emergências e ao desenvolvimento e implantação de soluções ligadas ao meio ambiente marinho e costeiro, principalmente para as indústrias marítima, portuária, pesqueira e de óleo e gás.
A OCP tem como principais atividades:
\begin{itemize}
\item Consultoria
\item Treinamento (com acreditação pelo emph{Nautical Institute})
\item Afretamento e operação de embarcações
\item Gerenciamento e resposta a emergências
\item Fornecimento de tripulações especializadas
\item \emph{Marine Survey} (oceanografia, geofísica e geotécnica)
\end{itemize}

Para realização destes serviços a empresa possui um total de 15 barcos, sendo:
\item 5 para contenção de derramamentos;
\item 5 para monitoramento ambiental;
\item 2 para abastecimento de óleo;
\item 3 ainda sem contrato de operação.

No que diz respeito ao item Gerenciamento e resposta a emergências, a empresa conta com um enorme variedade de navios e barcos destinados a tal fim, sendo um deles objeto de análise do presente estudo.

\section{Caracterização do Embarcação}
A embarcação escolhida para análise é a mais antiga da frota. A motivação para esta escolha se deve ao vasto histórico de não-conformidades quanto a fatores de segurança do trabalho evidenciados no decorrer das atividades desta embarcação desde seu primeiro contrato.

A embarcação em análise foi construída em LLLLL no ano de LLLL